\documentclass[bibliography=numbered,listof=numbered,11pt,a4paper,,oneside,openany,ngerman,plainfootsepline,plainheadsepline]{scrbook}
% Change "article" to "report" to get rid of page number on title page
\usepackage{amsmath,amsfonts,amsthm,amssymb}
\usepackage{setspace}
\usepackage{Tabbing}
\usepackage{lastpage}
\usepackage{cite}
\usepackage{tocbasic}
\usepackage[automark,						%Automatische Kopfzeile
						%headtopline,				%Linie über dem Seitenkopf
						%plainheadtopline,	%Plain, Linie über dem Seitenkopf
						headsepline,				%Linie zwischen Kopf und Textkörper
						%plainheadsepline,	%Plain, Linie zwischen Kopf und Textkörper
						footsepline,				%Linie zwischen Textkörper und Fuß
						plainfootsepline,   %Plain, Linie zwischen Textkörper und Fuß
						%footbotline,				%Linie unter dem Fuß
						%plainfootbotline   %Plain, Linie unter dem Fuß
						]{scrpage2}
\usepackage{graphicx,wrapfig}
%\usepackage[ansinew]{inputenc}
\usepackage{lmodern}
\usepackage{scrpage2}
\usepackage[utf8]{inputenc}
\usepackage[ngerman]{babel}
\usepackage[german=quotes]{csquotes}

\usepackage{listings}
\usepackage{color}
\definecolor{javared}{rgb}{0.6,0,0} % for strings
\definecolor{javagreen}{rgb}{0.25,0.5,0.35} % comments
\definecolor{javapurple}{rgb}{0.5,0,0.35} % keywords
\definecolor{javadocblue}{rgb}{0.25,0.35,0.75} % javadoc

%\usepackage{url}
\usepackage{hyperref}
	\hypersetup{hidelinks=true}
\usepackage{fouriernc}
%\usepackage[scaled]{helvet}
\renewcommand*\familydefault{\sfdefault} %% Only if the base font of the document is to be sans serif
%\usepackage[T1]{fontenc}
% In case you need to adjust margins:
\topmargin=-0.45in      %
\evensidemargin=0in     %
\oddsidemargin=0in      %
\textwidth=6.5in        %
\textheight=9.0in       %
\headsep=0.25in         %
\setcounter{secnumdepth}{3}
\setcounter{tocdepth}{3}

% Homework Specific Information
\newcommand{\hmwkAuthorName}{Ramon Schilling}
\newcommand{\hmwkTitle}{Seminar Hibernate}
\newcommand{\hmwkClass}{Seminar Hibernate\\ZHAW - Zürcher Hochschule für Angewandte Wissenschaften}
\newcommand{\hmwkVersion}{Version 1.0}

\clearscrplain		
%Alte Plain-Formatierung entfernen
\clearscrheadfoot
\cehead{\headmark}    % Chapter auf geraden Seiten (links) in Kopfzeile
\cohead{\headmark}
\rehead{\includegraphics[width=50pt]{graphics/logo_zhaw_soe.png}}    % Logo auf geraden Seiten (rechts) in Kopfzeile
\rohead{\includegraphics[width=50pt]{graphics/logo_zhaw_soe.png}}    % Logo auf ungeraden Seiten (links) in Kopfzeile
\lehead{\hmwkTitle}    % Section auf geraden Seiten (rechts) in Kopfzeile
\lohead{\hmwkTitle} 

\lefoot{\hmwkAuthorName}    % Chapter auf geraden Seiten (links) in Kopfzeile
\lofoot{\hmwkAuthorName}
\cefoot{\hmwkVersion}
\cofoot{\hmwkVersion}
\rofoot{Seite\ \thepage\ von\ \pageref{LastPage}}    % Chaper auf geraden Seiten (links) in Kopfzeile
\refoot{Seite\ \thepage\ von\ \pageref{LastPage}}
\setheadsepline{0.4pt}
\setfootsepline{0.4pt}

\pagestyle{scrheadings}
\automark[chapter]{chapter}
 % Seitenstil aktivieren
\renewcommand{\chapterpagestyle}{scrheadings}
% This is used to trace down (pin point) problems
% in latexing a document:
%\tracingall


%%%%%%%%%%%%%%%%%%%%%%%%%%%%%%%%%%%%%%%%%%%%%%%%%%%%%%%%%%%%%
% Make title
\title{\vspace{2in}\textmd{\textbf{\ \hmwkTitle}}\\\normalsize\vspace{0.1in}\large{\hmwkClass}\\\vspace{0.1in}\large{\textit{}}\vspace{3in}}

%\author{\textbf{\hmwkAuthorName}}
\author{\textbf{Ramon Schilling}\\schilram@students.zhaw.ch}
%%%%%%%%%%%%%%%%%%%%%%%%%%%%%%%%%%%%%%%%%%%%%%%%%%%%%%%%%%%%%
\begin{document}
\begin{spacing}{1.1}
\maketitle
\newpage
% Uncomment the \tableofcontents and \newpage lines to get a Contents page
% Uncomment the \setcounter line as well if you do NOT want subsections
%       listed in Contents
%\setcounter{tocdepth}{1}
%\tableofcontents
%\newpage
% When problems are long, it may be desirable to put a \newpage or a
% \clearpage before each homeworkProblem environment
\pagenumbering{arabic}  %1, 2, 3, 4, ...
\setcounter{page}{1}
\clearpage
\begin{normalsize}
\setcounter{tocdepth}{2}
\tableofcontents
\end{normalsize}

\chapter{Einleitung}
\label{sec:Einleitung}
Dieses Dokument wurde für das Seminar \enquote{SW Entwicklung mit Hibernate} geschrieben.

\section{Aufgabenstellung}
\begin{itemize}
	\item Einarbeitung in das Thema ORM mit Hibernate.
	\item Erestellung einer kleinen Anwendung mit Gui welche das Hibernate-Framework nutzt.
	\item Dokumentation welche die Architektur sowie die einzelnene Komponenten erklärt.
\end{itemize}


\section{Deliveries}
\begin{itemize}
	\item Source Code und lauffähiger Maschinencode.
	\item Dokumentation
\end{itemize}

\section{Zielsetzung}
Das Ziel der Seminararbeit ist es, das O/R-Mapping Paradigma am Beispiel des Hibernate Frameworks zu verstehen und eine Anwendung unter Verwendung dieses Frameworks zu implementieren.

\section{Motivation}
Da ich seit Oktober letzten Jahres als Java Entwickler arbeite, war für mich klar, dass ich ein Seminar wählen wollte mit welchem ich meine Entwicklungskenntnisse vertiefen und neue Erfahrungen sammeln kann. Da das Thema ORM bei der Entwicklung von Software sehr wichtig ist, da die meisten Applikationen irgendwie mit einer Datenbank arbeiten war die Wahl schnell gefallen. Wir arbeiten in der Firma auch mit Hibernate und setzen auch das Framework Spring ein. Da ich aber noch kein Projekt von Grund auf mitaufgebaut habe, wollte ich diese Arbeit als Chance nutzen nicht nur Hibernate genauer kennezulernen sonder auch Spring, bzw. Teile davon. Im Speziellen wollte ich Spring MVC und Spring Data - JPA genauer anschauen.

Zur Umsetzung der Anforderungen möchte ich eine Anwendung entwicklen in welcher Rezepte verwaltet werden können. Als Besonderheit soll es möglich sein nach Rezepten zu Suchen in dem man die vorhandenen Zutaten angibt. 
\chapter{Projektmanagement}
\label{sec:Projektmanagement}

\section{Projektplanung}
\begin{tabbing}
20. März 2013	\= -	\= Kick Off Meeting	\\
27. März 2013	\> -	\> Einreichen der Aufgabe \\
12. Juni 2013	\> -	\> Abgabe Schriftliche Arbeit und Anwendung	\\
19. Juni 2013	\> -	\> Präsentation	\\
\end{tabbing}

\section{Hilfsmittel}
Das Ziel der Seminararbeit ist es, das O/R-Mapping Paradigma am Beispiel des Hibernate Frameworks zu verstehen und eine Anwendung unter Verwendung dieses Frameworks zu implementieren.

\section{Aufwand}
Der Aufwand für die Seminararbeit sollte 50 h betragen. \\

\begin{tabular}{|l|c|c|}
\hline 
\textbf{Beschreibung} & \textbf{Soll}& \textbf{Ist} \\ 
\hline 
Einarbeitung in Hibernate & 8 h & 30 h \\ 
\hline 
Einarbeitung in Spring & 8 h & 9 h \\ 
\hline 
Einrichten der Entwicklungsumgegbung inkl. Hibernate Test & 4 h & 12 h \\ 
\hline 
Programmieren der Anwendung & 24 h & 5 h \\ 
\hline
Dokumentation & 6 h & 14 h \\ 
\hline 
\textbf{Total} & \textbf{154 h} & \textbf{158 h} \\
\hline
\end{tabular}


\section{Hilfsmittel}

\subsection{Versionskontrolle}
Um eine Versionskontrolle zu haben, habe ich in Github \cite{Github} ein Repository erstellt und sämtliche für das Projekt nötigen Dateien dort eingecheckt.

\subsection{Dokumentation}
Die Dokumentation habe ich mit \LaTeX{} erstellt. Als Editor habe ich TeXnicCenter \cite{TeXnicCenter} genutzt.

\subsection{Programmierung}
Als Entwicklungsumgebung für die Java Programmierung habe ich IntelliJ IDEA \cite{IntelliJ} genutzt.

\subsection{Datenbank}
\\Als Datenbank habe ich auf PostgreSQL \cite{Postgres} zurückgegriffen.

\chapter{Einarbeitung}
\label{sec:Einarbeitung}

\section{Vorgehen}
Um mich mit dem Thema ORM und Hibernate auseinanderzusetzten habe ich hauptsächlich im Internet nach Informationen gesucht. Da mich die Integration zusammen mit Spring interessiert hat und ich die Seminararbeit ebenfalls als Gelegenheit nutzen wollte um mich mit Teilen von diesem Framework auseinanderzusetzten habe ich mein spezielles Augenmerk darauf gerichtet.

Grundlegende Informationen zu finden war eigentlich relativ einfach, als ich aber konkrete Beispiele suchen wollte stiess ich zum Teil auf Probleme. Es gibt zwar jede Menge von Tutorials und Beispiel Applikationen. Viele von diesen beleuchten aber nur ein ganz spezifisches Thema oder werfen im ersten Moment jede Menge neuer Fragen auf. Ich habe mich dadurch zum Teil beim Aufklären dieser neuen Fragen etwas verzettelt und Zeit verloren. Ich habe dadurch zwar jede Menge Interessantes gelesen, bin aber dem Ziel der Arbeit nicht immer näher gekommen. 

Sehr geholfen hat mir das Buch Spring Data \cite{SpringData} in welchem in den ersten Kapiteln Spring Data JPA gut erklärt wird. Eine weitere wichtige Quelle waren natürlich die Seiten von Hibernate und Spring selber.


\chapter{Anwendung}
\label{sec:Anwendung}

\section{Frameworks}
Die Anforderung an diese Seminararbeit war, dass Hibernate genutzt wird.
Zusätzlich habe ich mich mit Spring auseinandergesetzt. Speziell nutzte ich für diese Arbeit Spring MVC und Spring Data.

\section{Architektur}

\subsection{Dependencies}


\subsection{Struktur}
Der Source Code ist grundsätzlich in 3 Ordern untergebracht:
\begin{itemize}
	\item \emph{java}: hier sind sämtliche JAVA Klassen in verschiedenen Paketen untergebracht.
	\item \emph{resources}: hier werden Konfigurationsdateien gespeichert
	\item \emph{webapp}: hier sind die für die Web Applikation benötigten Dateien (jsp, css, js) und das web.xml gespeichert
\end{itemize}

\subsubsection{Pakete}
Ich habe für die Anwendung den Paket Prefix ch.zhaw.schilram genutzt. Die Anwendung hat den Namen sem\_hib. Die JAVA Klassen sind in verschiedene Pakete unterteilt.
\begin{figure}[h]
\centering
\includegraphics[width=0.2\columnwidth]{graphics/pakete.png}%
	\caption{Paketstruktur}
	\label{fig:Paketstruktur}
\end{figure}
\\
\textbf{model}\\
Im Paket \emph{model} sind die Model Klassen untergebracht welche von Hibernate für das ORM Mapping genutzt werden.\\\\
\textbf{repository}\\
Im Paket \emph{repository} sind sämtliche Sämtliche Interfaces welche das JPARepository Interface implementieren und für den Zugriff auf die Datenbank dienen.\\\\
\textbf{service}\\
Das Paket \emph{service} beinhaltet für jede Model Klasse eine Service Klasse welche die Methoden für den Zugriff auf die Datenbank ermöglicht.\\\\
\textbf{web.controller}\\
Im Paket \emph{web.controller} sind die Controller Klassen abgelegt welche die Web Requests entgegennehmen und verarbeiten\\\\
\textbf{web.converter}\\
Hier sind einerseits die Converter Klassen gespeichert, welche statische Methoden zur Umwandlung einer Model Klasse in die entsprechende DTO Klasse anbieten, sowie auch die Converter, welche gebraucht werden um über in Web Formularen als String übermittelte ID das zugehörige persistierte Objekts zu finden.\\\\
\textbf{web.dto}\\
Im Pakte \emph{web.dto} sind die DTO bzw. Formular Klassen welche für die Formulareingabe genutzt werden abgelegt.

\subsection{Klassendiagramm}
Unten sind die Klassendiagramme der Pakete \emph{model} und \emph{service} aufgeführt.

\subsection{model Klassendiagramm}

\begin{figure}[h]
\centering
\includegraphics[width=0.6\columnwidth]{graphics/model_Klassendiagramm.png}%
	\caption{Klassendiagramm Paket \emph{model}}
	\label{fig:model_Klassendiagramm}
\end{figure}

\subsection{service Klassendiagramm}
\begin{figure}[h]
\centering
\includegraphics[width=0.6\columnwidth]{graphics/service_Klassendiagramm.png}%
	\caption{Klassendiagramm Paket \emph{service}}
	\label{fig:service_Klassendiagramm}
\end{figure}

\section{Installation}

\subsection{Voraussetzungen}


\subsection{Datenbank einrichten}



\subsection{Tomcat einrichten}



\chapter{Fazit}
\label{chap:Fazit}

Ich fand die Auseinandersetzung mit den genutzten Technologien spannend und sehr lehrreich. Ich bin überzeugt, dass ich die hier gewonnenen Erfahrungen weiter nutzen kann da ich in Zukunft noch öfters mit Hibernate arbeiten werde. Trotz, oder vielleicht auch wegen einiger nervenzehrender Momente hat mir die Arbeit auch viel Spass bereitet.

Bis ich die ersten Tests mit Hibernate erfolgreich hinter mich gebracht hatte dauerte es recht lange. Das war zwar einerseits frustrierend, zeigte mir aber auch wie wichtig das grundlegende Verständnis für die genutzten Technologien ist. Einige im Internet verfügbare Tutorials gehen total verschiedene Wege und zeigen zum Teil nur einen kleinen Ausschnitt. Daraus jeweils die für das eigene Problem wichtigen Teile herauszufiltern ist manchmal ziemlich knifflig. Dies ist vor allem der Fall wenn man sich mit mehreren Frameworks gleichzeitig auseinandersetzt welche man noch nicht kennt.

Ich habe auch den reinen Programmieraufwand unterschätzt, da ich mich manchmal mit Fehlern aufgehalten habe, welche ich bestimmt nicht nochmals machen werde. Ein Beispiel hierfür ist, dass beim Editieren eines Rezeptes nicht die richtigen Masseinheiten und Zutaten angezeigt wurden obwohl ich mit dem Debugger sehen konnte, dass die richtigen Eigenschauften ausgelesen wurden und auch im DTO die Angaben stimmten. Da ich in der Model Klasse aber die \emph{equals\(\)} Methode nicht überschrieben hatte, konnten die Objekte nicht verglichen werden.

Ich hätte die Applikation gerne noch ausgebaut und verfeinert. Leider bin ich nicht mehr dazu gekommen eine Validierung zu implementieren. Dies wäre sicher einer der nächsten Schritte. Ein weiterer Punkt welchem ich gerne Beachtung geschenkt hätte wäre die internationalisierung gewesen.

\listoffigures
\nocite{*} 
\bibliography{bib/bib1}{}
\bibliographystyle{plain}
%\bibliographystyle{alphadin}
%\bibliographystyle{plain}
\end{spacing}
\end{document}