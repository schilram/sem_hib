\chapter{Einleitung}
\label{chap:Einleitung}
Dieses Dokument wurde für das Seminar \enquote{SW Entwicklung mit Hibernate} geschrieben.

\section{Aufgabenstellung}
\begin{itemize}
	\item Einarbeitung in das Thema ORM mit Hibernate \cite{Hibernate}.
	\item Erestellung einer kleinen Anwendung mit Gui welche das Hibernate-Framework nutzt.
	\item Dokumentation welche die Architektur sowie die einzelnene Komponenten erklärt.
\end{itemize}


\section{Deliveries}
\begin{itemize}
	\item Source Code und lauffähiger Maschinencode.
	\item Dokumentation
\end{itemize}

\section{Zielsetzung}
Das Ziel der Seminararbeit ist es, das O/R-Mapping Paradigma am Beispiel des Hibernate Frameworks zu verstehen und eine Anwendung unter Verwendung dieses Frameworks zu implementieren.

\section{Motivation}
Da ich seit Oktober letzten Jahres als Java Entwickler arbeite, war für mich klar, dass ich ein Seminar wählen wollte mit welchem ich meine Entwicklungskenntnisse vertiefen und neue Erfahrungen sammeln kann. Da das Thema ORM bei der Entwicklung von Software sehr wichtig ist, da die meisten Applikationen irgendwie mit einer Datenbank arbeiten war die Wahl schnell gefallen. Wir arbeiten in der Firma auch mit Hibernate und setzen auch das Framework Spring \cite{Spring} ein. Da ich aber noch kein Projekt von Grund auf mitaufgebaut habe, wollte ich diese Arbeit als Chance nutzen nicht nur Hibernate genauer kennezulernen sonder auch Spring, bzw. Teile davon. Im Speziellen wollte ich Spring MVC und Spring Data genauer anschauen.

Zur Umsetzung der Anforderungen möchte ich eine Anwendung entwicklen in welcher Rezepte verwaltet werden können. Als Besonderheit soll es möglich sein nach Rezepten zu Suchen in dem man die vorhandenen Zutaten angibt. 