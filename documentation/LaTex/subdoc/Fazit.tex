\chapter{Fazit}
\label{chap:Fazit}

Ich fand die Auseinandersetzung mit den genutzten Technologien spannend und sehr lehrreich. Ich bin überzeugt, dass ich die hier gewonnenen Erfahrungen weiter nutzen kann da ich in Zukunft noch öfters mit Hibernate arbeiten werde. Trotz, oder vielleicht auch wegen einiger nervenzehrender Momente hat mir die Arbeit auch viel Spass bereitet.

Bis ich die ersten Tests mit Hibernate erfolgreich hinter mich gebracht hatte dauerte es recht lange. Das war zwar einerseits frustrierend, zeigte mir aber auch wie wichtig das grundlegende Verständnis für die genutzten Technologien ist. Einige im Internet verfügbare Tutorials gehen total verschiedene Wege und zeigen zum Teil nur einen kleinen Ausschnitt. Daraus jeweils die für das eigene Problem wichtigen Teile herauszufiltern ist manchmal ziemlich knifflig. Dies ist vor allem der Fall wenn man sich mit mehreren Frameworks gleichzeitig auseinandersetzt welche man noch nicht kennt.

Ich habe auch den reinen Programmieraufwand unterschätzt, da ich mich manchmal mit Fehlern aufgehalten habe, welche ich bestimmt nicht nochmals machen werde. Ein Beispiel hierfür ist, dass beim Editieren eines Rezeptes nicht die richtigen Masseinheiten und Zutaten angezeigt wurden obwohl ich mit dem Debugger sehen konnte, dass die richtigen Eigenschauften ausgelesen wurden und auch im DTO die Angaben stimmten. Da ich in der Model Klasse aber die \emph{equals\(\)} Methode nicht überschrieben hatte, konnten die Objekte nicht verglichen werden.

Ich hätte die Applikation gerne noch ausgebaut und verfeinert. Leider bin ich nicht mehr dazu gekommen eine Validierung zu implementieren. Dies wäre sicher einer der nächsten Schritte. Ein weiterer Punkt welchem ich gerne Beachtung geschenkt hätte wäre die internationalisierung gewesen.