\chapter{Projektmanagement}
\label{chap:Projektmanagement}

\section{Projektplanung}
\begin{tabbing}
20. März 2013	\= -	\= Kick Off Meeting	\\
27. März 2013	\> -	\> Einreichen der Aufgabe \\
12. Juni 2013	\> -	\> Abgabe Schriftliche Arbeit und Anwendung	\\
19. Juni 2013	\> -	\> Präsentation	\\
\end{tabbing}


\section{Aufwand}
Der Aufwand für die Seminararbeit sollte 50 h betragen. \\

\begin{tabular}{|l|c|c|}
\hline 
\textbf{Beschreibung} & \textbf{Soll}& \textbf{Ist} \\ 
\hline 
Einarbeitung in Hibernate & 4 h & 2 h \\ 
\hline 
Einarbeitung in Spring & 4 h & 2 h \\ 
\hline 
Einrichten der Entwicklungsumgegbung inkl. Hibernate Test & 4 h & 8 h \\ 
\hline 
Programmieren der Anwendung & 28 h & 32 h \\ 
\hline
Dokumentation & 8 h & 8 h \\ 
\hline 
\textbf{Total} & \textbf{48 h} & \textbf{52 h} \\
\hline
\end{tabular}
\\\\
Die Aufwandschätzung fiel mir relativ schwer, was sich auch darin zeigt, dass die Soll und Ist Zeiten nicht übereinstimmen. Ich habe für die theoretische Einarbeitung in Hibernate und Spring weniger Zeit aufwendete als geplant um schneller mit der Umsetzung zu beginnen. Bis ich dann aber alles soweit konfiguriert hatte, dass das Object-Relation-Mapping funktionierte hatte ich viel mehr Zeit gebraucht als geplant. Beim Programmieren habe ich ebenfalls z.T. sehr viel Zeit gebraucht um Fehler zu finden welche ich mit etwas mehr Erfahrung wohl gar nicht erst gemacht hätte.

\section{Hilfsmittel}

\subsection{Versionskontrolle}
Um eine Versionskontrolle zu haben, habe ich in Github \cite{Github} ein Repository erstellt und sämtliche für das Projekt nötigen Dateien dort eingecheckt.

\subsection{Dokumentation}
Die Dokumentation habe ich mit \LaTeX{} erstellt. Als Editor habe ich TeXnicCenter \cite{TeXnicCenter} genutzt.

\subsection{Programmierung}
Als Entwicklungsumgebung für die Java Programmierung habe ich IntelliJ IDEA \cite{IntelliJ} genutzt.

\subsection{Datenbank}
Als Datenbank habe ich auf PostgreSQL \cite{Postgres} zurückgegriffen.
