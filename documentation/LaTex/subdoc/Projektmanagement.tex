\chapter{Projektmanagement}
\label{sec:Projektmanagement}

\section{Projektplanung}
\begin{tabbing}
20. März 2013	\= -	\= Kick Off Meeting	\\
27. März 2013	\> -	\> Einreichen der Aufgabe \\
12. Juni 2013	\> -	\> Abgabe Schriftliche Arbeit und Anwendung	\\
19. Juni 2013	\> -	\> Präsentation	\\
\end{tabbing}

\section{Hilfsmittel}
Das Ziel der Seminararbeit ist es, das O/R-Mapping Paradigma am Beispiel des Hibernate Frameworks zu verstehen und eine Anwendung unter Verwendung dieses Frameworks zu implementieren.

\section{Aufwand}
Der Aufwand für die Seminararbeit sollte 50 h betragen. \\

\begin{tabular}{|l|c|c|}
\hline 
\textbf{Beschreibung} & \textbf{Soll}& \textbf{Ist} \\ 
\hline 
Einarbeitung in Hibernate & 8 h & 30 h \\ 
\hline 
Einarbeitung in Spring & 8 h & 9 h \\ 
\hline 
Einrichten der Entwicklungsumgegbung inkl. Hibernate Test & 4 h & 12 h \\ 
\hline 
Programmieren der Anwendung & 24 h & 5 h \\ 
\hline
Dokumentation & 6 h & 14 h \\ 
\hline 
\textbf{Total} & \textbf{154 h} & \textbf{158 h} \\
\hline
\end{tabular}


\section{Hilfsmittel}

\subsection{Versionskontrolle}
Um eine Versionskontrolle zu haben, habe ich in Github \cite{Github} ein Repository erstellt und sämtliche für das Projekt nötigen Dateien dort eingecheckt.

\subsection{Dokumentation}
Die Dokumentation habe ich mit \LaTeX{} erstellt. Als Editor habe ich TeXnicCenter \cite{TeXnicCenter} genutzt.

\subsection{Programmierung}
Als Entwicklungsumgebung für die Java Programmierung habe ich IntelliJ IDEA \cite{IntelliJ} genutzt.

\subsection{Datenbank}
\\Als Datenbank habe ich auf PostgreSQL \cite{Postgres} zurückgegriffen.
